\section{Analyse fonctionnelle}

	Cette section est destinée à justifier les choix de conception de ce projet afin de remplir le cahier des charges. Certains choix seront inspirés directement de solutions existantes listées dans la section précédente.
	
	La section suivante, sera directement liée à celle-ci car elle expliquera comment ces choix seront techniquement mis en œuvre lors de la phase d'implémentation.

	\subsection{Langage de programmation}
	%todo expliquer en quoi la syntaxe est intuitive
		Afin de garantir la facilité d'utilisation de notre générateur de contenu statique, notre choix de langage de programmation s'est tourné vers \textit{Python 3}. En effet, ce langage possède une syntaxe suffisamment flexible pour nous permettre de proposer à notre tour une syntaxe intuitive pour l'utilisateur.
	
	\subsection{Abstraction des données}
		Un des points forts de notre générateur de contenu statique est de pouvoir abstraire les données qu'il traite. Grâce à ce mécanisme, l'utilisateur est capable de manipuler une grande diversité de type de donnée sans avoir à adapter ses manipulations. Dit autrement, cela permet à l'utilisateur de travailler de la même manière avec plusieurs types de données différents sans avoir à adapter ses requêtes. Cela donne également la possibilité de manipuler ensemble des données  syntaxiquement différentes mais sémantiquement les mêmes. Dans ce cas, pour parler des données traitées, nous utiliserons le mot "\textbf{item}" introduit dans la section 2.1.
		
	\subsection{Configuration}
	%todo a qui profite la prog. modulaire ?
		Un autre point fort de notre générateur de contenu statique est qu'il est facilement configurable. En effet, il est possible de configurer son répertoire de travail ainsi que le moteur de rendu utilisé pour les \textit{templates}. Mais cela reste une fonctionnalité assez classique. L'originalité de notre système de configuration vient du fait qu'il est également possible d'y ajouter des modules \textit{Python 3} en plus des modules de bases que nous proposons. Cela permet à l'utilisateur d'y ajouter des fonctionnalités qu'il aurait lui-même développer pour que notre générateur réponde au mieux à ses besoins.
		

	\subsection{Règles de génération}
	%todo expliquer le principe général des règles 
		Pour ce qui est de l'utilisation en elle-même de notre générateur de contenu statique, nous avons pensé à un système de règles relativement intuitif.
		Ces règles sont divisées en plusieurs parties appelées \textbf{champs}. Chaque champs (ou ensemble de champs) représente un paramètre du générateur de contenu statique.	L'explication du fonctionnement de ces règles est approfondies dans la section suivantes. Ces règles permettent également à l'utilisateur de créer des variables qu'il peut ensuite utilisées dans le \textit{template}. Ces règles sont contenues dans un fichier qui sera passé en paramètre à notre programme qui les exécutera une par une. Chacune de ces règles symbolisera un des trois cas d'utilisation exposé en Section 2: \textit{AllToOne, OneToAll} et \textit{AlltoMany}.
		
	\subsection{Transformation des données}
	
		Une fonctionnalité importante de notre générateur de contenu statique est la possibilité d'appliquer des transformations aux données en cours de traitement. Cela peut par exemple permettre à l'utilisateur de formater certaines données avant de les injecter dans le \textit{template}. Afin de faciliter ce processus pour l'utilisateur, nous mettons aussi à sa disposition des méta-fonctions. Ces fonctions sont au nombre de trois: \textbf{map}, \textbf{reduce}, \textbf{filter}. 
		
		\textit{Map} est une méta-fonction qui permet d'appliquer une fonction à chaque \textit{item} d'un ensemble d'\textit{items}. Le résultat est donc un nouvel ensemble d'\textit{items} dont chaque  \textit{item} est le résultat de l'application de la fonction sur un \textit{item} de l'ensemble initial.
		
		\textit{Reduce} est une méta-fonction qui applique une fonction sur un ensemble d'\textit{items} deux par deux. L'un des deux items passé à cette fonction est le résultat obtenu à partir de la paire d'\textit{items} précédente. Le résultat est donc accumulé jusqu'à ce que la fonction ait été appliquée sur tout les \textit{items}. Le résultat de cette méta-onction est donc un unique \textit{item}.
		
		\textit{Filter} est une méta-fonction qui applique une fonction booléenne à chaque membre d'un ensemble d'\textit{item}. Le résultat de cette méta-fonction est un nouvel ensemble d'\textit{items} contenant uniquement les \textit{items} dont le résultat de la fonction est \texttt{True}.
		
		Ces trois méta-fonctions permettent à l'utilisateur de transformer les données en cours de traitement avec une très grande flexibilité.
	
	\subsection{Ajout de métadonnées}
	
	\subsection{Système de fichiers \textit{logs}}
	
		Un système de fichiers \textit{logs} permettrait de faciliter la correction de certaines règles lorsque les fichiers en sorties ne correspondent pas à ce que l'utilisateur attendait. Ces fichiers \textit{logs} n'auront comme utilité que de lister l'historique des règles exécutées pas à pas. Ce système viserait à atteindre l'objectif de simplicité d'utilisation -et d'aide à l'utilisation dans ce cas-ci- mentionné dans le cahier des charges en Section 2.