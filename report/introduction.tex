\section{Introduction}

	Dans le cadre du cours de projet et lecture et rédactions scientifiques, il nous est demandé de réaliser un projet d'envergure relativement importante. Dans notre cas, il s'agit d'implémenter un générateur dynamique de contenu statique. Ce rapport a pour but de vous présenter l'avancement de ce projet.\\
	
	Dans un premier temps, nous dresserons le cahier des charges de ce projet. Cela consistera à lister les fonctionnalités nécessaires à respecter afin que ce projet soit mené à bien. Nous illustrerons également quelques cas d'utilisations afin de clarifier le cahier des charges.\\
	
	Ensuite, nous listerons rapidement quelques générateurs de contenu statique existants. Nous exhiberons leurs différences ainsi que leurs fonctionnalités.\\
	
	Troisièmement, nous dresserons une liste de l'ensemble des fonctionnalités que devra posséder notre générateur de contenu statique. Nous en profiterons également pour le comparer aux autres générateurs présentés en Section 3. Cela nous permettra donc de justifier la nécessité de ce projet.\\
	
	Enfin, nous parlerons des techniques mises en place pour répondre aux besoins présentés en Section 4. En d'autres termes, cette section détaillera l'approche qui sera utilisée lors du développement de notre générateur de contenu statique.

	