\documentclass[12pt,a4paper]{memoire-umons}

\usepackage[utf8]{inputenc}
\usepackage[T1]{fontenc}
\usepackage[francais]{babel}
\usepackage{amssymb,amsmath,amsthm}
\usepackage{hyperref}% hyperliens dans le PDF, pas pour impression

\graphicspath{{./figures/}}

\title{Générateur Dynamique de Contenu Static}
\author{Maxime \textsc{De Wolf}}
\date{2017--2018}
\directeur{Decan Alexandre}
\service{Sciences Informatiques}
\rapporteurs{}
\discipline{informatiques}


%%%%%%%%%%%%%%%%%%%%%%%%%%%%%%%%%%%%%%%%%%%%%%%%%%%%%%%%%%%%%%%%%%%%%%%%
%% Vos macros


%%%%%%%%%%%%%%%%%%%%%%%%%%%%%%%%%%%%%%%%%%%%%%%%%%%%%%%%%%%%%%%%%%%%%%%%

% Compile uniquement certains morceaux sans perdre les références
% automatiques et la table des matières des parties déjà compilées :
%\includeonly{introduction,chapitre1}

\begin{document}
% Éventuellement utiliser l'environnement « preface » pour avoir une
% numérotation des pages en chiffres romains.
\tableofcontents

\chapter{Introduction}

	
\section{Etat de l'art}
	%todo justifier le choix des référence
	Cette section détaille, de manière non-exhaustive, l'existence d'autres générateurs de contenu statique similaires à ce projet. Leurs fonctionnalités principales seront présentées afin d'être comparées avec notre propre générateur.\\
	
	Ces générateurs de contenu statique ont été choisis grâce à StaticGen \cite{StaticGen} qui établie un classement d'un grand nombre de ces générateurs sur base de leur popularité sur GitHub. Parmi tous ceux qui y sont listés, nous en avons choisis trois comme étalons pour notre projet. Il s'agit de Jekyll, Pellican et Lektor.\\
	
	Nous pouvons expliquer ces choix comme suit: Jekyll est le générateur \textit{open-source} le plus populaire. Il s'agit donc d'un bon point de comparaison. Ensuite, Pellican est le générateur le plus populaire écrit en Python. Comme notre projet est également réalisé en Python, nous trouvons qu'il s'agit également d'une comparaison intéressante. Enfin, Lektor est le quatrième générateur le plus populaire réalisé en Python. Il offre cependant une plus grande souplesse d'utilisation que Pellican. Comme la souplesse d'utilisation est un critère important pour notre projet, nous avons jugé bon de l'ajouter à notre comparaison.
	
	\subsection*{Pelican}
	Pelican \cite{Pelican} est un générateur de site web statique dont les principales fonctionnalités sont:
	\begin{itemize}
		\item Le support de \textit{pages} (e.g. "Contact", ...)
		\item Le support d'\textit{articles} (e.g. posts d'un blog)
		\item La régénération rapide de fichiers grâce à un système de caches et d'écriture sélective.
		\item La gestion de thèmes créés à partir de \textit{templates Jinja2}
		\item La publication d'articles dans plusieurs langues
	\end{itemize}
	
	\subsection*{Lektor}
	Lektor \cite{Lektor} est également un générateur de site web statique. Ses principales fonctionnalités sont les suivantes:
	
	\begin{itemize}
		\item Un système de construction intelligent qui ne reconstruit que le \\contenu qui a été modifié
		\item Un outil graphique qui permet la modification de pages sans toucher au code source
		\item Utilisation de système de \textit{templates Jinja2} pour le rendu du contenu
		\item Un outil permettant la création relativement simple de sites web multilingues
	\end{itemize}
	
	On peut donc en conclure que c'est un outil assez similaire à Pelican sauf que Lektor propose un outil graphique pour l'édition de contenu.
	
	\subsection*{Jekyll}
	Enfin, Jekyll \cite{Jekyll} est le plus connu des générateurs de sites web statiques \textit{open source}. Il est écrit en \textit{Ruby} à la différence de Pelican et Lektor, écrits en \textit{Python}. Voici ses principales fonctionnalités:
	
	\begin{itemize}
		\item Utilisation de \textit{templates Liquid}
		\item Support de contenu de type \textit{pages} (e.g "Contact", "Accueil", ...) et \textit{articles} (e.g. posts d'un blog)
		\item Lancement d'un serveur local pour observer le rendu graphique des fichiers générés\\
	\end{itemize}
	
	%todo conclure l'état de l'art
	Malgré leurs différences, chacun de ces générateurs de sites web statiques fonctionnent selon le cas d'utilisation \textit{AllToOne} (voir Figure \ref{fig:ManyToOne}). Ce cas est très utilisé dans le cas de sites web de type blog d'où sa popularité.
\section{Cahier des charges}

	Cette section a pour rôle de présenter le cahier des charges de ce projet. Nous introduirons également le concept de générateur de contenu static. La seconde partie de cette section listera quelques cas d'utilisation afin d'illustrer le fonctionnement souhaité par ce projet. Notez que les fonctionnalités en elles-même seront détaillées dans la Section 4.\\
	
	Le cahier des charges de ce projet est volontairement resté assez flou. En effet, les seules consignes données étaient de concevoir un générateur de contenu static qui donnait à l'utilisateur une grande souplesse d'utilisation. La simplicité d'utilisation du générateur est également un critère important. Afin de clarifier ce que veut dire "souplesse d'utilisation", quelques cas d'utilisations seront illustrer dans cette section.\\
	
	
	\subsection{Générateur de contenu static}
	
		Un générateur de contenu static sert -comme son nom l'indique- à générer du contenu. Il sera qualifié de "static" si ce contenu ainsi généré est prêt à l'emploi et ne nécessite plus de transformation afin d'être utile. A savoir que les générateurs de contenu static sont très utilisés pour produire des sites web statics. Celui-ci aura une vocation un peu plus générale car il permettra de générer à peu près n'importe quel type de contenu static. Leur utilisation est souvent lié au concept de \textit{templates} qui permettent de séparer les informations de la mise en page en elle-même. La Figure \ref{fig:use_of_generator} montre une utilisation type de ce genre de générateur.\\
		
		\begin{figure}
			\begin{center}
				\begin{tikzpicture}
				\coordinate (Data) at (-3, 0);
				\coordinate (Template) at (-1, 1);
				\coordinate (Content) at (2, 0);
				\coordinate (Center) at (-1, 0);
				
				\draw (-4, 0) circle (1) ;
				\draw (-4, 0) node{$\text{Données}$};
				\draw[>=latex, ->] (Data) -- (Center);
				
				\draw[rounded corners]  (2, -1) rectangle(4, 1);
				\draw (3, 0) node{$\text{Contenu}$};
				\draw[>=latex, ->] (Center) --(Content);
				
				\draw(-2, 1) rectangle(0, 3);
				\draw (-1, 2) node{$\text{Template}$};
				\draw[>=latex, ->]  (Template) -- (Center);
			\end{tikzpicture}
			\caption{Utilisation type d'un générateur de contenu static}
			\label{fig:use_of_generator}
			\end{center}
		\end{figure}

	\subsection{Cas d'utilisations}
	
		Nous distinguons principalement trois cas d'utilisations propres aux générateurs de contenu statique. Chacun d'entre eux sera illustré à l'aide d'une figure. Nous appellerons ces trois cas comme suit: \textit{OneToAll}, \textit{AllToOne} et \textit{AllToMany}.\\
		
		\subsubsection*{OneToAll}
		
			La cas \textit{OneToAll} désigne l'utilisation d'un générateur de contenu static afin de générer plusieurs fichiers en sorties à partir d'un seul fichier en entré. Cela correspond, par exemple, à afficher un produit par page à partir d'un fichier contenant l'ensemble des produits disponibles. Cela se traduit par la Figure \ref{fig:OneToAll} en considérant que \textbf{n} est le nombre de produits contenus dans le fichier.
			
			\begin{figure}
				\begin{center}
					\begin{tikzpicture}
					\coordinate (Data) at (-3, 0);
					\coordinate (Template) at (-1, 1);
					\coordinate (Content) at (2, 0);
					\coordinate (Center) at (-1, 0);
					
					\draw (-4, 0) circle (1) ;
					\draw (-4, 0) node{$\text{Données}$};
					\draw[>=latex, ->] (Data) -- (Center) node[near start, above] {$1$};
					
					\draw[rounded corners]  (2, -1) rectangle(4, 1);
					\draw (3, 0) node{$\text{Contenu}$};
					\draw[>=latex, ->] (Center) --(Content) node[near end, above] {$n$};
					
					\draw(-2, 1) rectangle(0, 3);
					\draw (-1, 2) node{$\text{Template}$};
					\draw[>=latex, ->]  (Template) -- (Center) node[midway, right]{1};
					\end{tikzpicture}
					\caption{Représentation du cas \textbf{OneToAll}.}
					\label{fig:OneToAll}
				\end{center}
			\end{figure}
		
		\subsubsection*{AllToOne}
		
			Le second cas, \textit{AllToOne}, est le cas inverse de \textit{OneToAll}. En effet, ce cas désigne l'utilisation d'un générateur de contenu static pour générer un seul fichier à partir de plusieurs. En pratique, cela consiste par exemple à afficher l'intégralité des posts d'un blog sur une seule page, chaque page étant symbolisée par un fichier. Cela donne la Figure \ref{fig:AllToOne} où \textbf{n} est le nombre de fichiers contenant un post.
		
			\begin{figure}
				\begin{center}
					\begin{tikzpicture}
					\coordinate (Data) at (-3, 0);
					\coordinate (Template) at (-1, 1);
					\coordinate (Content) at (2, 0);
					\coordinate (Center) at (-1, 0);
					
					\draw (-4, 0) circle (1) ;
					\draw (-4, 0) node{$\text{Données}$};
					\draw[>=latex, ->] (Data) -- (Center) node[near start, above] {$n$};
					
					\draw[rounded corners]  (2, -1) rectangle(4, 1);
					\draw (3, 0) node{$\text{Contenu}$};
					\draw[>=latex, ->] (Center) --(Content) node[near end, above] {$1$};
					
					\draw(-2, 1) rectangle(0, 3);
					\draw (-1, 2) node{$\text{Template}$};
					\draw[>=latex, ->]  (Template) -- (Center) node[midway, right]{1};
					\end{tikzpicture}
					\caption{Représentation du cas \textbf{AllToOne}.}
					\label{fig:AllToOne}
				\end{center}
			\end{figure}
			
		\subsubsection*{AllToMany}
			
			Le dernier cas est celui que nous appellerons \textit{AllToMany}. Ce cas est en fait une généralisation des deux cas précédant. Effectivement, \textit{AllToMany} désigne l'utilisation d'un générateur de contenu static afin de générer plusieurs fichiers en sortis à partir de plusieurs fichier d'entrés. Mis en contexte, il désigne par exemple la génération de fichiers contenant chacun plusieurs posts à partir de fichiers contenant chacun un post. La Figure \ref{fig:AllToMany} permet une vision plus claire de ce cas où \textbf{n} désigne le nombre de fichier contenant chacun un post et \textbf{m} le nombre de pages (resp. fichiers) contenant un certains nombre de posts.\\
			
			\begin{figure}
				\begin{center}
					\begin{tikzpicture}
					\coordinate (Data) at (-3, 0);
					\coordinate (Template) at (-1, 1);
					\coordinate (Content) at (2, 0);
					\coordinate (Center) at (-1, 0);
					
					\draw (-4, 0) circle (1) ;
					\draw (-4, 0) node{$\text{Données}$};
					\draw[>=latex, ->] (Data) -- (Center) node[near start, above] {$n$};
					
					\draw[rounded corners]  (2, -1) rectangle(4, 1);
					\draw (3, 0) node{$\text{Contenu}$};
					\draw[>=latex, ->] (Center) --(Content) node[near end, above] {$m$};
					
					\draw(-2, 1) rectangle(0, 3);
					\draw (-1, 2) node{$\text{Template}$};
					\draw[>=latex, ->]  (Template) -- (Center) node[midway, right]{$1$};
					\end{tikzpicture}
					\caption{Représentation du cas \textbf{AllToMany}.}
					\label{fig:AllToMany}
				\end{center}
			\end{figure}
		
		\subsection{Cas non-couverts}
			Un lecteur attentif aura remarqué que nous ne discutons pas de la multiplicité au niveau de l'application des \textit{templates}. Pour un soucis de complétude, nous les listerons dans cette section. Ils feront éventuellement parti de l'implémentation finale si le temps nous le permet. Nous appellerons ces cas \textit{MultiTemplatesIn} et \textit{MultiTemplatesOut}.
			
			 Notez que ces cas d'utilisations seraient réalisables sans implémentation additionnelle. Malheureusement cela impliquerait quelques contraintes pour l'utilisateur ce qui nuirait à la souplesse d'utilisation de notre générateur de contenu static. Nous discuterons de cela plus avant dans la Section 5.
			
			\subsubsection*{MultiTemplatesIn}
				Ce cas d'utilisation implique  d'utiliser un générateur de contenu static dans le but de produire un fichier contenant plusieurs fois les mêmes données mais chaque fois agencées par un \textit{template} différent. Nous n'avons pas trouvé d'exemple pratique mais cela pourrait servir à comparer plusieurs \textit{templates} entre eux. La Figure \ref{fig:MultiTemplatesIn} illustre ce cas plus clairement, où \textbf{n} est le nombre de \textit{templates} à appliquer sur le fichier de données.
				
				\begin{figure}
					\begin{center}
						\begin{tikzpicture}
						\coordinate (Data) at (-3, 0);
						\coordinate (Template) at (-1, 1);
						\coordinate (Content) at (2, 0);
						\coordinate (Center) at (-1, 0);
						
						\draw (-4, 0) circle (1) ;
						\draw (-4, 0) node{$\text{Données}$};
						\draw[>=latex, ->] (Data) -- (Center) node[near start, above] {$1$};
						
						\draw[rounded corners]  (2, -1) rectangle(4, 1);
						\draw (3, 0) node{$\text{Contenu}$};
						\draw[>=latex, ->] (Center) --(Content) node[near end, above] {$1$};
						
						\draw(-2, 1) rectangle(0, 3);
						\draw (-1, 2) node{$\text{Template}$};
						\draw[>=latex, ->]  (Template) -- (Center) node[midway, right]{$n$};
						\end{tikzpicture}
						\caption{Représentation du cas \textbf{MultiTemplatesIn}.}
						\label{fig:MultiTemplatesIn}
					\end{center}
				\end{figure}
				
			
			\subsubsection*{MultiTemplatesOut}
				Ce cas d'utilisation est similaire au précédent, il implique également l'application de plusieurs \textit{templates} sur un même fichier de données. La différence vient du fait que \textit{MultiTemplatesOut} produit un fichier en sortie par \textit{template} appliqué. En pratique, cela veut par exemple dire que pour un fichier contenant les informations propres à un \textit{CV} on veuille générer un fichier \LaTeX et une page web. Ces deux contenus contiendraient les mêmes informations mais l'un pourra être imprimé (une fois le PDF produit) et l'autre sera mis à disposition sur Internet. Cela se traduit par la Figure \ref{fig:MultiTemplatesOut} où \textbf{n} est le nombre de \textit{templates} à appliquer sur le fichier.
				
				\begin{figure}
					\begin{center}
						\begin{tikzpicture}
						\coordinate (Data) at (-3, 0);
						\coordinate (Template) at (-1, 1);
						\coordinate (Content) at (2, 0);
						\coordinate (Center) at (-1, 0);
						
						\draw (-4, 0) circle (1) ;
						\draw (-4, 0) node{$\text{Données}$};
						\draw[>=latex, ->] (Data) -- (Center) node[near start, above] {$1$};
						
						\draw[rounded corners]  (2, -1) rectangle(4, 1);
						\draw (3, 0) node{$\text{Contenu}$};
						\draw[>=latex, ->] (Center) --(Content) node[near end, above] {$n$};
						
						\draw(-2, 1) rectangle(0, 3);
						\draw (-1, 2) node{$\text{Template}$};
						\draw[>=latex, ->]  (Template) -- (Center) node[midway, right]{$n$};
						\end{tikzpicture}
						\caption{Représentation du cas \textbf{MultiTemplatesOut}.}
						\label{fig:MultiTemplatesOut}
					\end{center}
				\end{figure}
				
	
\section{Terminologie}

	\begin{itemize}
		\item \textbf{Target:}
		\item \textbf{Template:}
		\item \textbf{Data:}
		\item \textbf{Output:}
		\item \textbf{Rule:} 
	\end{itemize}
\section{Format des données}
\section{Implémentation}

	\subsection{Configuration}
	
		\begin{itemize}
			\item input dir
			\item output dir
			\item Rulebackend, TemplateBackend, ...
			\item 
		\end{itemize}
	
	\subsection{Structure}

		Surcharger le \textit{RuleBackend} permet par exemple de changer la syntaxe.
\section{Conclusion}

% etc

% Si vous utilisez (conseillé) BibTeX pour votre bibliographie :
\bibliographystyle{acm}
\bibliography{memoire}% si le fichier BibTeX est memoire.bib

\end{document}
%%% Local Variables: 
%%% mode: latex
%%% TeX-master: t
%%% TeX-PDF-mode: t
%%% End: 
