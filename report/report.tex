\documentclass[12pt,a4paper]{memoire-umons}

\usepackage[utf8]{inputenc}
\usepackage[T1]{fontenc}
\usepackage[francais]{babel}
\usepackage{amssymb,amsmath,amsthm}
\usepackage{hyperref}% hyperliens dans le PDF, pas pour impression
\usepackage{xcolor}
\usepackage{tikz}
\usepackage{tcolorbox}

%------Notes------------
\newcounter{note}[section]
\newenvironment{note}[1][]
{	
	
	\refstepcounter{note}
	
	\begin{center}
		\begin{tcolorbox}[title=\textbf{Remarque \arabic{section}.\arabic{note}:}, colback=white, colbacktitle=white, coltitle=black]
			\begin{em}
			}
			{
			\end{em}
		\end{tcolorbox}	
	\end{center}

}
%----------------------


\graphicspath{{./figures/}}

\title{Générateur Dynamique de Contenu Statique}
\author{Maxime \textsc{De Wolf}}
\date{2017--2018}
\directeur{Decan Alexandre}
\service{Sciences Informatiques}
\rapporteurs{}
\discipline{informatiques}


%%%%%%%%%%%%%%%%%%%%%%%%%%%%%%%%%%%%%%%%%%%%%%%%%%%%%%%%%%%%%%%%%%%%%%%%
%% Vos macros


%%%%%%%%%%%%%%%%%%%%%%%%%%%%%%%%%%%%%%%%%%%%%%%%%%%%%%%%%%%%%%%%%%%%%%%%

% Compile uniquement certains morceaux sans perdre les références
% automatiques et la table des matières des parties déjà compilées :
%\includeonly{introduction,chapitre1}

\begin{document}
% Éventuellement utiliser l'environnement « preface » pour avoir une
% numérotation des pages en chiffres romains.
\tableofcontents
%todo préférer le présent au passé/futur
\section{Introduction}

	Dans le cadre du cours de projet et lecture et rédactions scientifiques, il nous est demandé de réaliser un projet d'envergure relativement importante. Dans notre cas, il s'agit d'implémenter un générateur dynamique de contenu statique. Ce rapport a pour but de vous présenter l'avancement de ce projet.\\
	
	Dans un premier temps, nous dresserons le cahier des charges de ce projet. Cela consistera à lister les fonctionnalités nécessaires à respecter afin que ce projet soit mené à bien. Nous illustrerons également quelques cas d'utilisations afin de clarifier le cahier des charges.\\
	
	Ensuite, nous listerons rapidement quelques générateurs de contenu statique existants. Nous exhiberons leurs différences ainsi que leurs fonctionnalités.\\
	
	Troisièmement, nous dresserons une liste de l'ensemble des fonctionnalités que devra posséder notre générateur de contenu statique. Nous en profiterons également pour le comparer aux autres générateurs présentés en Section 3. Cela nous permettra donc de justifier la nécessité de ce projet.\\
	
	Enfin, nous parlerons des techniques mises en place pour répondre aux besoins présentés en Section 4. En d'autres termes, cette section détaillera l'approche qui sera utilisée lors du développement de notre générateur de contenu statique.

	
\section{Cahier des charges}

	Cette section a pour rôle de présenter le cahier des charges de ce projet. Nous introduisons également le concept de générateur de contenu statique. La seconde partie de cette section liste quelques cas d'utilisation afin d'illustrer le fonctionnement souhaité par ce projet. Notez que les fonctionnalités en elles-même seront détaillées dans la Section 4.\\
	
	Le cahier des charges de ce projet est volontairement resté assez flou. En effet, les seules consignes données sont de concevoir un générateur de contenu statique qui donne à l'utilisateur une grande souplesse d'utilisation afin de pouvoirs effectuer des tâches diverses. Ce générateur de contenu statique doit aussi être simple à utiliser pour le "grand public" et ne doit donc pas faire appel à des connaissances en programmation. Afin de préciser ce que sont les "tâches diverses" que ce générateur doit être capable d'accomplir, quelques cas d'utilisations sont illustrés dans cette section.\\
	
	
	\subsection{Générateur de contenu statique}
	%todo insister sur le rôle, pas un générateur de site web, différence avec CMS ? Motivation générale du projet
		Nous introduisons ici les notations que nous allons utiliser tout au long de ce rapport. Cette sous-section contient également notre définition d'un générateur de contenu statique. 
	
		\begin{defn}
			Un \textbf{générateur de contenu} sert à lier des données à un \textit{template} (ou gabarit) afin de générer du contenu (document, site web, ...). Il sera qualifié de \textbf{statique} si ce contenu est identique peu importe le contexte dans lequel il est consulté. Les données, les \textit{templates} et les contenus sont appelés \textbf{paramètres} d'un générateur de contenu statique dans le reste de ce rapport.
		\end{defn}
		
		\begin{defn}
			Un \textbf{item} est la brique de base d'un paramètre. Il s'agit d'une abstraction de ce qu'un paramètre représente comme valeur. En pratique un \textit{item} peut être une page web, un fichier texte, une base de données, une structure de données, ...
		\end{defn}
		
		\begin{defn}
			 La \textbf{multiplicité} des paramètres d'un générateur de contenu statique est le nombre d'\textit{items} qu'ils représentent.
		\end{defn}
		
		\begin{note}
			Nous utilisons parfois les termes "générateur de contenu statique" et "générateur" de manière interchangeable afin d'éviter les répétitions et les lourdeurs d'écritures.
		\end{note}
		
		 A savoir que les générateurs de contenu statique sont très utilisés pour produire des sites web statiques. De ce cas, les \textit{CMS} (ou SGC) peuvent être considérés comme des générateurs de contenu statique. Ils offrent toutefois plus de fonctionnalités qu'un générateur classique mais cela s'écarte du cadre de ce projet. Cela n'est vrai que pour les \textit{CMS} qui ne génère que du contenu statique.\\
		 
		 Notre générateur de contenu statique a une vocation un peu plus générale car il permettra de générer à peu près n'importe quel type de contenu statique. Leur utilisation est souvent liée au concept de \textit{templates} qui permettent de séparer les informations de la mise en page en elle-même. La Figure \ref{fig:use_of_generator} montre une utilisation type de ce genre de générateur. L'Exemple \ref{exmpl:GDCS} illustre cela grâce à un cas simple et concret.\\
		
		\begin{figure}[h]
			\begin{center}
				\begin{tikzpicture}
				\coordinate (Data) at (-3, 0);
				\coordinate (Template) at (-1, 1);
				\coordinate (Content) at (2, 0);
				\coordinate (Center) at (-1, 0);
				
				\draw (-4, 0) circle (1) ;
				\draw (-4, 0) node{$\text{Données}$};
				\draw[>=latex, ->] (Data) -- (Center);
				
				\draw[rounded corners]  (2, -1) rectangle(4, 1);
				\draw (3, 0) node{$\text{Contenu}$};
				\draw[>=latex, ->] (Center) --(Content);
				
				\draw(-2, 1) rectangle(0, 3);
				\draw (-1, 2) node{$\text{Template}$};
				\draw[>=latex, ->]  (Template) -- (Center);
			\end{tikzpicture}
			\caption{Utilisation type d'un générateur de contenu statique}
			\label{fig:use_of_generator}
			\end{center}
		\end{figure}
		
		\begin{exmpl}
			\label{exmpl:GDCS}
			Les Figures \ref{exmpl:GDCS:data}, \ref{exmpl:GDCS:template}, \ref{exmpl:GDCS:content} représentent respectivement les données à utiliser, le \textit{template} à appliquer et le contenu ainsi généré. Nous pouvons ainsi constater que le générateur de contenu statique génère ce dernier en remplaçant les champs du \textit{template} par les valeurs contenues dans le fichier de données.
			\begin{figure}[!]
				\centering
				\lstinputlisting[inputencoding=utf8/latin1]{codeSample/example/players.json}
				\caption{Données au format JSON}
				\label{exmpl:GDCS:data}
			\end{figure}				
			\begin{figure}[!]
				\centering
				\lstinputlisting[inputencoding=utf8/latin1]{codeSample/example/template.j2}
				\caption{Template au format Jinja2}
				\label{exmpl:GDCS:template}
			\end{figure}%
			\begin{figure}[!]
				\centering
				\lstinputlisting[inputencoding=utf8/latin1]{codeSample/example/players.txt}
				\caption{Contenu au format \textit{txt}}
				\label{exmpl:GDCS:content}
			\end{figure}			
			
		\end{exmpl}

	\subsection{Cas d'utilisations}
	
		Nous distinguons principalement trois cas d'utilisations propres aux générateurs de contenu statique. Nous appelons ces trois cas comme suit: \textit{OneToMany}, \textit{ManyToOne} et \textit{ManyToMany}. En pratique, ces trois cas se distinguent par la multiplicités des paramètres du générateur de contenu statique. Chacun d'entre eux est illustré à l'aide d'une figure explicitant leurs multiplicités.\\
		
		Nous présentons également 2 cas d'utilisations supplémentaires qui ne sont pas directement réalisables grâce à notre générateur de contenu statique. Nous présentons donc aussi des techniques pour contourner ces limitations. Ces cas d'utilisations sont appelés \textit{ManyTemplatesIn} et \textit{ManyTemplatesOut}.\\
		
		\subsubsection*{OneToMany}
		
			La cas \textit{OneToMany} désigne l'utilisation d'un générateur de contenu statique en vue de générer plusieurs \textit{items} de contenus à partir d'un seul \textit{item} de données. Ce cas d'utilisation est illustré par la Figure \ref{fig:OneToMany} en considérant que \textbf{n} est la multiplicité des contenus à générer. L'Exemple \ref{exmpl:OneToMany} met en pratique ce cas d'utilisation.
			
			\begin{figure}[!]
				\begin{center}
					\begin{tikzpicture}
					\coordinate (Data) at (-3, 0);
					\coordinate (Template) at (-1, 1);
					\coordinate (Content) at (2, 0);
					\coordinate (Center) at (-1, 0);
					
					\draw (-4, 0) circle (1) ;
					\draw (-4, 0) node{$\text{Données}$};
					\draw[>=latex, ->] (Data) -- (Center) node[near start, above] {$1$};
					
					\draw[rounded corners]  (2, -1) rectangle(4, 1);
					\draw (3, 0) node{$\text{Contenu}$};
					\draw[>=latex, ->] (Center) --(Content) node[near end, above] {$n$};
					
					\draw(-2, 1) rectangle(0, 3);
					\draw (-1, 2) node{$\text{Template}$};
					\draw[>=latex, ->]  (Template) -- (Center) node[midway, right]{1};
					\end{tikzpicture}
					\caption{Représentation du cas \textbf{OneToMany}.}
					\label{fig:OneToMany}
				\end{center}
			\end{figure}
			
			\begin{exmpl}
				\label{exmpl:OneToMany}
				 Par exemple, pour un site de vente, cela correspond à générer autant de page web qu'il y a de produits, chaque page ne contenant les caractéristiques que d'un unique produit (\textit{items} de contenu). L'\textit{item} de données est ici représenté par une base de données contenant les caractéristiques de tous les produits. Dans cet exemple, le \textit{template} est appliqué à chaque produit individuellement pour générer une page web par produit.
			\end{exmpl}
		
		\subsubsection*{ManyToOne}
		
			Le second cas, \textit{ManyToOne}, est le cas inverse de \textit{OneToMany}. En effet, ce cas désigne l'utilisation d'un générateur de contenu statique qui génère un seul \textit{item} à partir de plusieurs. Cela donne la Figure \ref{fig:ManyToOne} où \textbf{n} est la multiplicité des données. L'Exemple \ref{exmpl:ManyToOne} met ce cas d'utilisation en pratique.
			
			\begin{exmpl}
				\label{exmpl:ManyToOne}
				Par exemple, pour un blog, cela correspond à générer une page web contenant l'ensemble de tous les posts (\textit{item} de contenu). Celle-ci serait générée à partir d'un ensemble de fichier contenant chacun un post (\textit{items} de données). Dans ce cas, le \textit{template} est appliqué une  fois sur l'ensemble des posts dans le but de générer une unique page web.
			\end{exmpl}
		
			\begin{figure}[!]
				\begin{center}
					\begin{tikzpicture}
					\coordinate (Data) at (-3, 0);
					\coordinate (Template) at (-1, 1);
					\coordinate (Content) at (2, 0);
					\coordinate (Center) at (-1, 0);
					
					\draw (-4, 0) circle (1) ;
					\draw (-4, 0) node{$\text{Données}$};
					\draw[>=latex, ->] (Data) -- (Center) node[near start, above] {$n$};
					
					\draw[rounded corners]  (2, -1) rectangle(4, 1);
					\draw (3, 0) node{$\text{Contenu}$};
					\draw[>=latex, ->] (Center) --(Content) node[near end, above] {$1$};
					
					\draw(-2, 1) rectangle(0, 3);
					\draw (-1, 2) node{$\text{Template}$};
					\draw[>=latex, ->]  (Template) -- (Center) node[midway, right]{1};
					\end{tikzpicture}
					\caption{Représentation du cas \textbf{ManyToOne}.}
					\label{fig:ManyToOne}
				\end{center}
			\end{figure}
			
		\subsubsection*{ManyToMany}
			
			Le dernier cas est celui que nous appellerons \textit{ManyToMany}. Ce cas est en fait une généralisation des deux cas précédents. Effectivement, \textit{ManyToMany} désigne l'utilisation d'un générateur de contenu statique en vue de générer plusieurs \textit{items} de contenu à partir de plusieurs \textit{items} de données. La Figure \ref{fig:ManyToMany} permet une vision plus claire de ce cas où \textbf{n} désigne la multiplicité des données et \textbf{m} la multiplicité des contenus. L'Exemple \ref{exmpl:ManyToMany} met en pratique ce cas d'utilisation.\\
			
			\begin{exmpl}
				\label{exmpl:ManyToMany}
				 Par exemple, pour un blog, cela correspond à produire un certains nombres de pages web (disons 100 pages) listant les posts 10 par 10 (\textit{items} de contenus) grâce à un ensemble de fichiers contenant chacun un post (\textit{items} de données). Ici, le \textit{template} est appliqué 100 fois sur des ensembles de 10 posts jusqu'à ce que tous les posts soient traités. Cela génère les 100 pages contenant chacune 10 posts.  
			\end{exmpl}
			
			\begin{figure}[!]
				\begin{center}
					\begin{tikzpicture}
					\coordinate (Data) at (-3, 0);
					\coordinate (Template) at (-1, 1);
					\coordinate (Content) at (2, 0);
					\coordinate (Center) at (-1, 0);
					
					\draw (-4, 0) circle (1) ;
					\draw (-4, 0) node{$\text{Données}$};
					\draw[>=latex, ->] (Data) -- (Center) node[near start, above] {$n$};
					
					\draw[rounded corners]  (2, -1) rectangle(4, 1);
					\draw (3, 0) node{$\text{Contenu}$};
					\draw[>=latex, ->] (Center) --(Content) node[near end, above] {$m$};
					
					\draw(-2, 1) rectangle(0, 3);
					\draw (-1, 2) node{$\text{Template}$};
					\draw[>=latex, ->]  (Template) -- (Center) node[midway, right]{$1$};
					\end{tikzpicture}
					\caption{Représentation du cas \textbf{ManyToMany}.}
					\label{fig:ManyToMany}
				\end{center}
			\end{figure}
		
		%todo expliquer pourquoi ils sont réalisables sans implémentation
		
		\subsubsection*{ManyTemplatesIn}
		%todo cas théorique, expliquer en montrant un exemple
			Ce cas d'utilisation implique  d'utiliser un générateur de contenu statique dans le but de produire un fichier contenant plusieurs fois les mêmes données mais agencées par un \textit{template} différent. Nous n'avons pas trouvé d'exemple pratique pour illustrer ce cas mais la Figure \ref{fig:ManyTemplatesIn:out} permet de visualiser le format de fichier générer dans ce cas. La Figure \ref{fig:ManyTemplatesIn} exhibe plus clairement les multiplicités de ce cas, où \textbf{n} est le nombre de \textit{templates} à appliquer sur le fichier de données.
			
			\begin{figure}[!]
				\begin{center}
					\begin{tikzpicture}
					\coordinate (Data) at (-3, 0);
					\coordinate (Template) at (-1, 1);
					\coordinate (Content) at (2, 0);
					\coordinate (Center) at (-1, 0);
					
					\draw (-4, 0) circle (1) ;
					\draw (-4, 0) node{$\text{Données}$};
					\draw[>=latex, ->] (Data) -- (Center) node[near start, above] {$1$};
					
					\draw[rounded corners]  (2, -1) rectangle(4, 1);
					\draw (3, 0) node{$\text{Contenu}$};
					\draw[>=latex, ->] (Center) --(Content) node[near end, above] {$1$};
					
					\draw(-2, 1) rectangle(0, 3);
					\draw (-1, 2) node{$\text{Template}$};
					\draw[>=latex, ->]  (Template) -- (Center) node[midway, right]{$n$};
					\end{tikzpicture}
					\caption{Représentation du cas \textbf{ManyTemplatesIn}.}
					\label{fig:ManyTemplatesIn}
				\end{center}
			\end{figure}
			
			
			\begin{figure}[!]
				\begin{center}
					\begin{tikzpicture}
					\coordinate (Data) at (-3, 0);
					\coordinate (Template) at (-1, 1);
					\coordinate (Content) at (2, 0);
					\coordinate (Center) at (-1, 0);
				
					
					\draw(0, 0) rectangle(4, 6);
					\draw (2, -0.5) node{$\text{Fichier}$};
					
					\draw(0.5, 4.5) rectangle(3.5, 5.5);
					\draw (2, 5) node{$\text{Template 1}$};
					
					\draw(0.5, 3) rectangle(3.5, 4);
					\draw (2, 3.5) node{$\text{Template 2}$};
					
					\draw[dashed] (2, 2.75) -- (2, 1.75);
					
					\draw(0.5, 0.5) rectangle(3.5, 1.5);
					\draw (2, 1) node{$\text{Template n}$};
					
					\end{tikzpicture}
					\caption{Représentation d'un fichier de sorti du cas \textbf{ManyTemplatesIn}.}
					\label{fig:ManyTemplatesIn:out}
				\end{center}
			\end{figure}
			
			\begin{note}
				Notre générateur de contenu statique ne permet pas de gérer ce cas directement. Néanmoins, il existe un moyen de contourner cette limitation. En effet, suivant le moteur de \textit{template} utilisé, il est possible de créer un \textit{template} qui en inclut d'autres. Cela permet donc de produire un comportement similaire à \textit{ManyTemplatesIn}.
			\end{note}
			
		
		\subsubsection*{ManyTemplatesOut}
		
			Ce cas d'utilisation est similaire au précédent, il implique également l'application de plusieurs \textit{templates} sur un même fichier de données. La différence vient du fait que \textit{ManyTemplatesOut} produit un fichier en sortie par \textit{template} appliqué. L'Exemple \ref{exmpl:ManyTemplatesOut} met ce cas d'utilisation en pratique. Cela se traduit par la Figure \ref{fig:ManyTemplatesOut} où \textbf{m} est le nombre de \textit{templates} à appliquer sur le fichier.
			
			\begin{exmpl}
				\label{exmpl:ManyTemplatesOut}
				En pratique, cela revient à dire que pour un fichier contenant les informations propres à un \textit{CV}, on veut générer un fichier \LaTeX et une page web. Ces deux contenus contiendraient les mêmes informations mais l'un pourra être imprimé (une fois le PDF produit) et l'autre sera mis à disposition sur Internet.
			\end{exmpl}
			
			\begin{note}
				Notre générateur de contenu statique ne donne pas à l'utilisateur le contrôle sur la multiplicité des \textit{templates}. Cependant, comme pour le cas \textit{ManyTemplatesIn}, il existe un moyen de contourner cette limitation afin de pouvoir réaliser le cas \textit{ManyTemplatesOut}. En effet, intuitivement, ce cas d'utilisation correspond à \textbf{m} cas d'utilisation \textit{ManyToOne} où \textbf{n} vaut 1. Ce cas d'utilisation est donc réalisable mais cela demande un travail supplémentaire de la part de l'utilisateur qui devra réaliser plusieurs cas \textit{ManyToOne} afin d'y parvenir.
			\end{note}
			
			\begin{figure}[!]
				\begin{center}
					\begin{tikzpicture}
					\coordinate (Data) at (-3, 0);
					\coordinate (Template) at (-1, 1);
					\coordinate (Content) at (2, 0);
					\coordinate (Center) at (-1, 0);
					
					\draw (-4, 0) circle (1) ;
					\draw (-4, 0) node{$\text{Données}$};
					\draw[>=latex, ->] (Data) -- (Center) node[near start, above] {$1$};
					
					\draw[rounded corners]  (2, -1) rectangle(4, 1);
					\draw (3, 0) node{$\text{Contenu}$};
					\draw[>=latex, ->] (Center) --(Content) node[near end, above] {$m$};
					
					\draw(-2, 1) rectangle(0, 3);
					\draw (-1, 2) node{$\text{Template}$};
					\draw[>=latex, ->]  (Template) -- (Center) node[midway, right]{$m$};
					\end{tikzpicture}
					\caption{Représentation du cas \textbf{ManyTemplatesOut}.}
					\label{fig:ManyTemplatesOut}
				\end{center}
			\end{figure}
				
	
\section{Etat de l'art}
	
	Cette section détaille, de manière non-exhaustive, l'existence d'autres générateurs de contenu statique \textit{similaires} à ce projet. Leurs fonctionnalités principales seront présentées afin d'être comparées avec notre propre générateur.
	
	\subsection*{Pelican}
		Pelican est un générateur de site web statique dont les principales fonctionnalités sont:
		\begin{itemize}
			\item Le support de \textit{pages} (e.g. "Contact", ...)
			\item Le support d'\textit{articles} (e.g. posts d'un blog)
			\item La régénération rapide de fichiers grâce à un système de caches et d'écriture sélective.
			\item La gestion de \textit{thèmes} créés à partir de \textit{template jinja2}.
		\end{itemize}
		
	\subsection*{Lektor}
		Lektor est également un générateur de site web statique. 

	\subsection{Cas d'utilisations}
\section{Analyse fonctionnelle}

	Cette section est destinée à justifier les choix de conception de ce projet afin de remplir le cahier des charges. Certains choix seront inspirés directement de solutions existantes listées dans la section précédente.
	
	La section suivante, sera directement liée à celle-ci car elle expliquera comment ces choix seront techniquement mis en œuvre lors de la phase d'implémentation.

	\subsection{Langage de programmation}
	%todo expliquer en quoi la syntaxe est intuitive
		Afin de garantir la facilité d'utilisation de notre générateur de contenu statique, notre choix de langage de programmation s'est tourné vers \textit{Python 3}. En effet, ce langage possède une syntaxe suffisamment flexible pour nous permettre de proposer à notre tour une syntaxe intuitive pour l'utilisateur.
		
	\subsection{Programmation modulaire}
	%todo a qui profite la prog. modulaire ?
		Pour ce qui est de la souplesse d'utilisation, nous appliquerons les concepts de la programmation modulaire. Cela permettra à l'utilisateur de modifier ou de créer facilement de nouvelles fonctionnalités pour notre générateur de contenu statique. Nous fournirons tout de même les fonctionnalités de bases nécessaires au bon fonctionnement du projet.

	\subsection{Règles}
	%todo expliquer le principe général des règles 
		Pour ce qui est de l'utilisation en elle-même de notre générateur de contenu statique, nous avons pensé à un système de règles relativement simple. Ces règles proposeront un système de variables pour permettre à l'utilisateur de réutiliser la même valeur dans plusieurs champs de la même règle. Ces règles seront contenues dans un fichier qui sera passé en paramètre à notre programme qui les exécutera une par une. Chacune de ces règles symbolisera un des trois cas d'utilisation exposé en Section 2: \textit{AllToOne, OneToAll} et \textit{AlltoMany}. De plus amples informations sur la syntaxe de ces règles seront détaillées dans la section suivante.
	
	\subsection{Système de fichiers \textit{logs}}
	
		Un système de fichiers \textit{logs} permettrait de faciliter la correction de certaines règles lorsque les fichiers en sorties ne correspondent pas à ce que l'utilisateur attendait. Ces fichiers \textit{logs} n'auront comme utilité que de lister l'historique des règles exécutées pas à pas. Ce système viserait à atteindre l'objectif de simplicité d'utilisation -et d'aide à l'utilisation dans ce cas-ci- mentionné dans le cahier des charges en Section 2.
\section{Analyse technique}

	\subsection{Règles}
		\begin{itemize}
			\item \textbf{Target:}
			\item \textbf{Template:}
			\item \textbf{Data:}
			\item \textbf{Output:}
			\item \textbf{Rule:} 
		\end{itemize}
	
	\subsection{Configuration}
	
		\begin{itemize}
			\item input dir
			\item output dir
			\item Rulebackend, TemplateBackend, ...
			\item 
		\end{itemize}
	
	\subsection{Structure}
	
	Surcharger le \textit{RuleBackend} permet par exemple de changer la syntaxe.
%\section{Conclusion}

	Nous avons mis au point un générateur de contenu statique doté d'une grande souplesse d'utilisation. En effet, il différent des autres générateurs de contenu statique de par sa capacité à permettre à l'utilisateur de réaliser une plus grande diversité de cas d'utilisation. Cette particularité lui permet également de ne pas se cantonner à la génération d'un seul type de contenu. Sa polyvalence est donc également une de ses forces.\\
	
	De plus, nous avons fait notre possible pour proposer à l'utilisateur la syntaxe la plus claire et la plus simple possible. En effet, les fichiers propres à la manipulation de notre générateur, comme les fichiers de configurations et de règles, sont au format \textit{YAML}, un format relativement simple à prendre en main. En plus de cela, nous avons pris soin à ce que la syntaxe des expressions utilisées dans les règles de génération soit la plus claire et la plus intuitive possible.\\
	
	Notre générateur de contenu statique donne également à l'utilisateur la possibilité de transformer les données chargées à la volée. Pour ce faire, nous mettons à disposition de l'utilisateur un ensemble de fonctions de base ainsi que trois méta-fonctions. Notre générateur donne aussi l'opportunité de traiter toutes sortes de fichiers de données sans devoir modifier les règles de génération. Cela est possible grâce aux \textit{loaders} qui se charge de l'abstraction de ces données.\\
	
	Enfin, si l'utilisateurs veut avoir accès à d'autres fonctions ou pouvoir charger des données d'un type non pris en charge, il peut écrire lui-même ses modules et ses \textit{loaders}. Il peut ensuite les encoder dans un fichier de configuration afin de pouvoir les utiliser dans les règles de génération.\\
	
	Même si nous avons tout de même constaté qu'il pouvait être amélioré relativement facilement, notre générateur de contenu statique respecte toute les exigences du cahier des charges et contient toutes les fonctionnalités présentée dans l'analyse fonctionnelle. Nous pouvons donc dire que le générateur que nous avons développé comporte toutes les fonctionnalités désirées à sa création.


% etc

% Si vous utilisez (conseillé) BibTeX pour votre bibliographie :
\bibliographystyle{acm}
\bibliography{memoire}% si le fichier BibTeX est memoire.bib

\end{document}
%%% Local Variables: 
%%% mode: latex
%%% TeX-master: t
%%% TeX-PDF-mode: t
%%% End: 
