\documentclass[12pt,a4paper]{memoire-umons}

\usepackage[utf8]{inputenc}
\usepackage[T1]{fontenc}
\usepackage[francais]{babel}
\usepackage{amssymb,amsmath,amsthm}
\usepackage{hyperref}% hyperliens dans le PDF, pas pour impression

\graphicspath{{./figures/}}

\title{Générateur Dynamique de Contenu Static}
\author{Maxime \textsc{De Wolf}}
\date{2017--2018}
\directeur{Decan Alexandre}
\service{Sciences Informatiques}
\rapporteurs{}
\discipline{informatiques}


%%%%%%%%%%%%%%%%%%%%%%%%%%%%%%%%%%%%%%%%%%%%%%%%%%%%%%%%%%%%%%%%%%%%%%%%
%% Vos macros


%%%%%%%%%%%%%%%%%%%%%%%%%%%%%%%%%%%%%%%%%%%%%%%%%%%%%%%%%%%%%%%%%%%%%%%%

% Compile uniquement certains morceaux sans perdre les références
% automatiques et la table des matières des parties déjà compilées :
%\includeonly{introduction,chapitre1}

\begin{document}
% Éventuellement utiliser l'environnement « preface » pour avoir une
% numérotation des pages en chiffres romains.
\tableofcontents

\section{Introduction}

	Dans le cadre du cours de projet et lecture et rédactions scientifiques, il nous est demandé de réaliser un projet d'envergure relativement importante. Dans notre cas, il s'agit d'implémenter un générateur dynamique de contenu statique. Ce rapport a pour but de vous présenter l'avancement de ce projet.\\
	
	Dans un premier temps, nous dresserons le cahier des charges de ce projet. Cela consistera à lister les fonctionnalités nécessaires à respecter afin que ce projet soit mené à bien. Nous illustrerons également quelques cas d'utilisations afin de clarifier le cahier des charges.\\
	
	Ensuite, nous listerons rapidement quelques générateurs de contenu statique existants. Nous exhiberons leurs différences ainsi que leurs fonctionnalités.\\
	
	Troisièmement, nous dresserons une liste de l'ensemble des fonctionnalités que devra posséder notre générateur de contenu statique. Nous en profiterons également pour le comparer aux autres générateurs présentés en Section 3. Cela nous permettra donc de justifier la nécessité de ce projet.\\
	
	Enfin, nous parlerons des techniques mises en place pour répondre aux besoins présentés en Section 4. En d'autres termes, cette section détaillera l'approche qui sera utilisée lors du développement de notre générateur de contenu statique.

	
\section{Etat de l'art}
	
	Cette section détaille, de manière non-exhaustive, l'existence d'autres générateurs de contenu statique \textit{similaires} à ce projet. Leurs fonctionnalités principales seront présentées afin d'être comparées avec notre propre générateur.
	
	\subsection*{Pelican}
		Pelican est un générateur de site web statique dont les principales fonctionnalités sont:
		\begin{itemize}
			\item Le support de \textit{pages} (e.g. "Contact", ...)
			\item Le support d'\textit{articles} (e.g. posts d'un blog)
			\item La régénération rapide de fichiers grâce à un système de caches et d'écriture sélective.
			\item La gestion de \textit{thèmes} créés à partir de \textit{template jinja2}.
		\end{itemize}
		
	\subsection*{Lektor}
		Lektor est également un générateur de site web statique. 

	\subsection{Cas d'utilisations}
% etc

% Si vous utilisez (conseillé) BibTeX pour votre bibliographie :
\bibliographystyle{acm}
\bibliography{memoire}% si le fichier BibTeX est memoire.bib

\end{document}
%%% Local Variables: 
%%% mode: latex
%%% TeX-master: t
%%% TeX-PDF-mode: t
%%% End: 
