\section{Etat de l'art}
	%todo justifier le choix des référence
	Cette section détaille, de manière non-exhaustive, l'existence d'autres générateurs de contenu statique similaires à ce projet. Leurs fonctionnalités principales seront présentées afin d'être comparées avec notre propre générateur.\\
	
	Ces générateurs de contenu statique ont été choisis grâce à StaticGen \cite{StaticGen} qui établie un classement d'un grand nombre de ces générateurs sur base de leur popularité sur GitHub. Parmi tous ceux qui y sont listés, nous en avons choisis trois comme étalons pour notre projet. Il s'agit de Jekyll, Pellican et Lektor.\\
	
	Nous pouvons expliquer ces choix comme suit: Jekyll est le générateur \textit{open-source} le plus populaire. Il s'agit donc d'un bon point de comparaison. Ensuite, Pellican est le générateur le plus populaire écrit en Python. Comme notre projet est également réalisé en Python, nous trouvons qu'il s'agit également d'une comparaison intéressante. Enfin, Lektor est le quatrième générateur le plus populaire réalisé en Python. Il offre cependant une plus grande souplesse d'utilisation que Pellican. Comme la souplesse d'utilisation est un critère important pour notre projet, nous avons jugé bon de l'ajouter à notre comparaison.
	
	\subsection*{Pelican}
	Pelican \cite{Pelican} est un générateur de site web statique dont les principales fonctionnalités sont:
	\begin{itemize}
		\item Le support de \textit{pages} (e.g. "Contact", ...)
		\item Le support d'\textit{articles} (e.g. posts d'un blog)
		\item La régénération rapide de fichiers grâce à un système de caches et d'écriture sélective.
		\item La gestion de thèmes créés à partir de \textit{templates Jinja2}
		\item La publication d'articles dans plusieurs langues
	\end{itemize}
	
	\subsection*{Lektor}
	Lektor \cite{Lektor} est également un générateur de site web statique. Ses principales fonctionnalités sont les suivantes:
	
	\begin{itemize}
		\item Un système de construction intelligent qui ne reconstruit que le \\contenu qui a été modifié
		\item Un outil graphique qui permet la modification de pages sans toucher au code source
		\item Utilisation de système de \textit{templates Jinja2} pour le rendu du contenu
		\item Un outil permettant la création relativement simple de sites web multilingues
	\end{itemize}
	
	On peut donc en conclure que c'est un outil assez similaire à Pelican sauf que Lektor propose un outil graphique pour l'édition de contenu.
	
	\subsection*{Jekyll}
	Enfin, Jekyll \cite{Jekyll} est le plus connu des générateurs de sites web statiques \textit{open source}. Il est écrit en \textit{Ruby} à la différence de Pelican et Lektor, écrits en \textit{Python}. Voici ses principales fonctionnalités:
	
	\begin{itemize}
		\item Utilisation de \textit{templates Liquid}
		\item Support de contenu de type \textit{pages} (e.g "Contact", "Accueil", ...) et \textit{articles} (e.g. posts d'un blog)
		\item Lancement d'un serveur local pour observer le rendu graphique des fichiers générés\\
	\end{itemize}
	
	%todo conclure l'état de l'art
	Malgré leurs différences, chacun de ces générateurs de sites web statiques fonctionnent selon le cas d'utilisation \textit{AllToOne} (voir Figure \ref{fig:ManyToOne}). Ce cas est très utilisé dans le cas de sites web de type blog d'où sa popularité.