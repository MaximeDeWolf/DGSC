\section{Analyse technique}
	Cette section est destinée à détailler les techniques mises en œuvres afin d'implémenter les différentes fonctionnalités décrites dans la section précédente. Le but de cette analyse technique est de montrer le raisonnement qui a conduit à la structure actuelle de notre générateur de contenu statique. Nous n'exposerons donc pas les détails d'implémentations. Si cela vous intéresse, le code source à disposition sur \url{https://github.com/MaximeDeWolf/DGSC/tree/master/code/src}.
	
	\subsection{Règles}
	%todo trouver un autre nom pour "target" qui suggère une entrée
	%todo mieux expliquer ce que désigne les champs d'une règle + exemple
	
	Une règle de génération est le moyen que nous mettons à disposition à l' utilisateur pour que celui-ci spécifie à notre générateur de contenu statique pour désigner le comportement que celui-ci doit adopter.
	Typiquement, une règle représente un cas d'utilisation. Il s'agit donc d'un des points les plus importants du projet. Cette sous-section explique donc en détail les différents éléments qui définissent ces règles ainsi que leur structure.
	
	\subsubsection*{Le format}
	
		Le format des fichiers de règles se doit de posséder deux qualités importantes. La première est d'être facilement compréhensible par l'utilisateur aussi bien lors de la lecture que de l'écriture. La deuxième est de pouvoir être analysé facilement par notre générateur de contenu statique dans le but d'y lire les informations nécessaire à son fonctionnement.
		
		Nous avons donc choisis le format \textit{YAML} pour stocker ces règles. Chaque fichier de règles contient donc une liste de règles. Ces règles contiennent chacune quatre champs qui eux-mêmes contiennent une expression. Ce sont ces expressions qui seront évaluées par notre générateur. La Figure \ref{fig:rule:format} illustre le format type d'un fichier de règles.
		
		\begin{figure}[!]
			\centering
			\lstinputlisting[inputencoding=utf8/latin1]{codeSample/ruleFormat.yaml}
			\caption{Format d'un fichier de règles de notre générateur}
			\label{fig:rule:format}
		\end{figure}
		
	\subsubsection*{Les expressions}
	
	\subsubsection*{Les champs}
		Ces règles possèdent quatre champs obligatoires: \textit{target}, \textit{template}, \textit{data} et \textit{output}. Comme expliqué dans la précédente section, chacun de ces champs (ou ensemble de champs) permet de  contrôler les paramètres de notre générateur. \textit{target} et \textit{output} représentent le contenudu générateur tandis que \textit{template} et \textit{data} représentent respectivement le \textit{template} et les données. Les particularités et fonctions de chacun de ces champs sont expliqués ci-dessous.
		
		
		\subsubsection*{Target}
			Le champ \textit{target} contient une expression destinée à être lue par notre générateur. Le résultat de cette expression sert à définir la multiplicité du paramètre "contenu" de la règle. Cela définit le nombre de fois que les autres champs de la règles courantes doivent être évalués. A chaque itération de ce processus un \textit{item} différent est stocké dans une variable appelée \textit{current}. Cette variable est ensuite utilisable dans les autres champs de la règle courante.
		
		\subsubsection*{Template}
		%todo clarifier quel moteur de template gère le cas MultiTemplatesIn
			Ce champs contient une expression destinée à être évaluée par notre générateur de contenu statique. Le résultat de cette expression à désigner le \textit{template} à utiliser sur les données chargées grâce au champ \textit{data}.\\
			
			La multiplicité de ce paramètre est volontairement bloquée à un. En effet, cela simplifie ainsi l'utilisation de notre générateur de contenu statique bien cela limite également souplesse d'utilisation. Cette limitation nous empêche de réaliser les cas d'utilisation \textbf{ManyTemplatesIn} et \textbf{ManyTemplateOut}. Nous avons toutefois montré un moyen de contourner cette limitation dans la Section 2.
		
		\subsubsection*{Data}
			Ce champs contient un ensemble de paires clés/valeurs. Chacune de ces valeurs est une expression destinée à être évaluée par notre générateur de contenu statique. Chaque clé de cette ensemble devient une variable dont la valeur est le résultat de son expression évaluée par notre générateur. Ces variables sont par la suite utilisables dans le \textit{template} sélectionné par le champ \textit{template}. La multiplicité de ce champs définit la multiplicité du paramètre "données" du cas d'utilisation. 
		
		\subsubsection*{Output}
			Ce champ contient une expression destinée à être évaluée par notre générateur de contenu statique. Le résultat de cette expression désigne le chemin vers le fichier où le résultat final de l'application du \textit{template} sur les données doit être écrit. Ce champs influence indirectement la multiplicité de paramètre "contenu" du cas d'utilisation. En effet, même si la multiplicité de ce paramètre est sensée être supérieure à un, fournir à tous ces \textit{items} de contenu le même fichier de sortie entraîne l'écrasement de ceux-ci pour au final ne contenir que le dernier \textit{item} généré.
	
	
	\subsection{Configuration}
	
		Le fichier de configuration est stocké au format \textit{YAML} pour garantir à l'utilisateur une simplicité de lecture est d'écriture. Ce format permet également à notre générateur de lire facilement. De plus, ce format permet de hiérarchiser les différentes options. Cela nous permet par exemple de séparer les options propres au \textit{template} de celle propre à notre générateur en lui-même. Ces deux catégories s'appellent respectivement \textbf{TEMPLATE} et \textbf{PRODUCTION}.
		
		Ce fichier de configuration permet à l'utilisateur de spécifier quelques options à notre générateur de contenu statique. Ces options sont les suivantes: répertoire de travail en entrée, les modules à charger, les \textit{loaders}, le répertoire de \textit{templates} et le moteur de \textit{template}. Les trois première font partie de la catégorie PRODUCTION alors que les deux dernières font partie de la catégorie \textit{TEMPLATE}.\\
		
		Le répertoire de travail en entrée, représenté par le nom \textbf{WORKING\_DIR} dans le fichier, spécifie le chemin menant au répertoire contenant les fichiers de données. Cela permet à l'utilisateur de ne pas avoir à écrire de longs chemins de fichier dans le champ \textit{data}. Sa valeur par défaut est le répertoire courant: ".".
		
		Les modules à charger, représentés par le nom \textbf{MODULES}, énumère les chemins menant aux modules Python écris par un tiers. Ces modules sont ensuite chargés par notre générateur de contenu statique afin que ceux-ci soient utilisables dans les expressions des règles de génération. Par défaut cette liste contient uniquement tous les modules de bases de notre générateur de contenu statique.
		
		Les \textit{loaders}, représentés par le nom \textbf{LOADERS}, listent les chemins menant aux modules Python écris par un tiers. Ces modules sont ensuite chargés par notre générateur de contenus statique afin que ceux-ci soient utilisables. La fonction \textbf{load} utilise ensuite ces modules lorsqu'elle est appelée dans le but de pouvoir charger des fichiers sur base de leur extension. Par défaut, \textit{LOADERS} contient au moins les \textit{loaders} de base c'est à dire ceux destiné à charger des fichiers \textit{YAML} et \textit{Json}.\\
		
		Le répertoire de \textit{templates}, représenté par le nom \textbf{DIR}, est le chemin menant au répertoire contenant les fichiers \textit{template}. Cela permet à l'utilisateur d'éviter d'écrire de trop long chemins de fichier dans les expressions destinées à désigner le chemin d'un \textit{template}. Sa valeur par défaut est ".".
		
		Le moteur de \textit{template}, appelé \textbf{BACKEND}, est le chemin du module Python qui permet à notre générateur de contenu d'utiliser un moteur de \textit{template} installé par l'utilisateur. Ce module se doit de respecter une certaine interface pour que cela soit possible. Il doit implémenter les méthodes suivantes: \textbf{initEnvironment}, \textbf{loadTemplate}, \textbf{loadData} et \textbf{render}. La méthode \textit{initEnvironment} permet d'initialiser le répertoire de \textit{template} pour le moteur de \textit{template}. Les méthodes \textit{loadTemplate} et \textit{loadData} permettent respectivement de charger un \textit{template} et de charger des \textit{données}. Enfin, \textit{render} permet d'appliquer le \textit{template} chargé sur les données et retourne le résultat final. La valeur par défaut de cette option est "renderers/jinja\_renderer.py".\\
		
		Le chemin d'un fichier de configuration peut être passé en paramètre en ligne de commande lors de l'évaluation d'un ou plusieurs fichiers de règles. Cela permet à l'utilisateur d'avoir un contrôle précis sur la configuration qui est appliquée à un ou plusieurs fichiers de règles.
		
		Si aucun fichier de configuration n'est passé en paramètre à la ligne de commande, notre générateur chargera par défaut la configuration appelée "config.yaml".
		
		Si cette configuration n'existe pas non plus, notre générateur utilisera alors la configuration par défaut. La Figure \ref{fig:config:default} montre cette configuration par défaut.
		
		Un fichier de configuration n'a pas besoin de spécifier toutes les options. Celles qui ne sont pas mentionnées prennent automatiquement les valeurs par défaut.	Si une configuration spécifie les options \textit{MODULES} et \textit{LOADERS} alors ils sont automatiquement agrémentés de leurs valeurs par défauts respectives.
		
		\begin{figure}[!]
			\centering
			\lstinputlisting[inputencoding=utf8/latin1]{codeSample/defaultConfig.yaml}
			\caption{Configuration par défaut de notre générateur}
			\label{fig:config:default}
		\end{figure}
		
		\begin{note}
			En pratique, la configuration par défaut présentée à la Figure \ref{fig:config:default} est encodée directement dans un dictionnaire Python. Cependant, nous nous jugeons préférable de vous la montrer au format \textit{YAML} pour qu'elle serve également à illustrer le format que doit avoir un fichier de configuration.
		\end{note}
		
	\subsection{Syntaxe}
	
	\subsection{Modules}
		
		